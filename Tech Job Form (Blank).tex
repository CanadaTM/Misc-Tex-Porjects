\documentclass[12pt]{article}
\usepackage[utf8]{inputenc}
\usepackage{hyperref}
\usepackage{xcolor}
\usepackage{graphicx}
\usepackage{tipa}
\usepackage{multirow}
\usepackage[object=vectorian]{pgfornament}

\newcommand{\sectionlinetwo}[2]{%
    \nointerlineskip \vspace{-0.5\baselineskip}\hspace{\fill}
    {\resizebox{0.5\linewidth}{1.2ex}
        {\pgfornament[color = #1]{#2}
        }}%
    \hspace{\fill}
    \par\nointerlineskip \vspace{.5\baselineskip}
}
\newcommand{\bWidth}{1}
\newcommand\blfootnote[1]{%
    \begingroup
    \renewcommand\thefootnote{}\footnote{#1}%
    \addtocounter{footnote}{-1}%
    \endgroup
}
\newlength{\sectionSpace}
\setlength{\sectionSpace}{20px}

\definecolor{hyperlinkblue}{RGB}{49,62,130}
\definecolor{jumpred}{RGB}{211,11,40}
\definecolor{textfieldbg}{rgb}{.85,.85,.85}
\definecolor{borderorange}{RGB}{255,189,168}
\definecolor{hintgray}{RGB}{150,150,150}
\definecolor{hinthighlight}{RGB}{255,150,150}

\hypersetup{
    colorlinks=true,
    linkcolor=jumpred,
    urlcolor=hyperlinkblue,
    linktoc=all,
    pdftitle={Tech Job Form},
    bookmarks=true,
}
\def\LayoutCheckField#1#2{#2 #1}

\begin{document}

\begin{titlepage}
    \addtocontents{toc}{\protect\hypertarget{toc}{}}
    \tableofcontents
    \vspace{3.5cm}

    \section*{General Information\blfootnote{\hyperlink{toc}{Back to Top}}}
    \rule[10pt]{\textwidth}{2pt}
    \addcontentsline{toc}{section}{General Information}
    \label{sec:General}

    \noindent
    \vspace{2px}
    \begin{Form}
        \textbf{\emph{Customer's Name:}}\newline
        \TextField[
            name = customer.name,
            borderwidth = \bWidth,
            bordercolor = borderorange,
            backgroundcolor = textfieldbg,
            width = \textwidth,
        ]{}

        \vspace{\sectionSpace}
        \noindent
        \textbf{\emph{Account Number:}}
        \TextField[
            name = account.number,
            borderwidth = \bWidth,
            bordercolor = borderorange,
            backgroundcolor = textfieldbg,
            width = 6.5em,
        ]{}
        \ChoiceMenu[
            name = account.type,
            borderwidth = \bWidth,
            bordercolor = borderorange,
            backgroundcolor = textfieldbg,
            radio = true,
        ]{\textbf{\emph{Type:}}}{Residential, Commercial}

        \vspace{\sectionSpace}
        \textbf{\emph{Systems:}}

        \vspace{3px}
        \noindent
        \makeatletter
        \@for%
        \system:={Security System,Fire System,Camera System,Access System}%
        \do{%
            \vspace{3px}%
            \CheckBox[%
                name = \system,%
                borderwidth = \bWidth,%
                bordercolor = borderorange,%
                backgroundcolor = textfieldbg%
            ]{ \system}%
            \newline%
        }
        \makeatother
    \end{Form}
\end{titlepage}

\section*{Security System\blfootnote{\hyperlink{toc}{Back to Top}}}
\rule[10pt]{\textwidth}{2pt}
\addcontentsline{toc}{section}{Security System}
\label{sec:Security}

\noindent
\begin{Form}
    \TextField[
        name = security.paneltype,
        borderwidth = \bWidth,
        bordercolor = borderorange,
        backgroundcolor = textfieldbg,
        width = 7em,
        height = .8\baselineskip
    ]{\textbf{\emph{Panel Type:}}}

    \vspace{\sectionSpace}
    \textbf{\emph{Location:}}\newline
    \vspace{2px}
    \TextField[
        name = security.panellocation,
        borderwidth = \bWidth,
        bordercolor = borderorange,
        backgroundcolor = textfieldbg,
        width = \textwidth,
        height = .8\baselineskip
    ]{}

    \textbf{\emph{Transformer Location:}}\newline
    \vspace{2px}
    \TextField[
        name = security.transformerlocation,
        borderwidth = \bWidth,
        bordercolor = borderorange,
        backgroundcolor = textfieldbg,
        width = \textwidth,
        height = .8\baselineskip
    ]{}

    \vspace{\sectionSpace}
    \emph{\textbf{Cell Unit}}

    \vspace{2px}
    \noindent
    \TextField[
        name = security.cellid,
        maxlen = 7,
        borderwidth = \bWidth,
        bordercolor = borderorange,
        backgroundcolor = textfieldbg,
        width = 4em,
        height = .8\baselineskip
    ]{\emph{\textbf{I.D:}}}

    \vspace{5px}
    \underline{Or}

    \vspace{5px}
    \noindent
    \makeatletter
    \@for%
    \macbox:={1,2,3,4,5,6}%
    \do{%
        \ifnum\macbox=1%
            \TextField[%
                name = security.macbox\macbox,%
                maxlen = 2,%
                borderwidth = \bWidth,%
                bordercolor = borderorange,%
                backgroundcolor = textfieldbg,%
                width = 2em,%
                height = .8\baselineskip%
            ]{\emph{\textbf{M.A.C:}}}
        \else%
            \TextField[%
                name = security.macbox\macbox,%
                maxlen = 2,%
                borderwidth = \bWidth,%
                bordercolor = borderorange,%
                backgroundcolor = textfieldbg,%
                width = 2em,%
                height = .8\baselineskip%
            ]{\emph{\textbf{:}}}
        \fi%
    }%
    \makeatother
    \hspace{.1em}
    \TextField[
        name = security.crc,
        maxlen = 4,
        borderwidth = \bWidth,
        bordercolor = borderorange,
        backgroundcolor = textfieldbg,
        width = 7.5em,
        height = .8\baselineskip
    ]{\textbf{\emph{C.R.C:}}}

    \vspace{5px}
    \underline{And maybe}

    \vspace{5px}
    \noindent
    \TextField[%
        name = security.honeywellcity,%
        maxlen = 2,%
        borderwidth = \bWidth,%
        bordercolor = borderorange,%
        backgroundcolor = textfieldbg,%
        width = 2em,%
        height = .8\baselineskip%
    ]{\emph{\textbf{City:}}}
    \emph{\textbf{\textemdash}}
    \TextField[%
        name = security.honeywellcsid,%
        maxlen = 2,%
        borderwidth = \bWidth,%
        bordercolor = borderorange,%
        backgroundcolor = textfieldbg,%
        width = 2em,%
        height = .8\baselineskip%
    ]{\emph{\textbf{CSID:}}}
    \emph{\textbf{\textemdash}}
    \TextField[%
        name = security.honeywellaccount,%
        maxlen = 4,%
        borderwidth = \bWidth,%
        bordercolor = borderorange,%
        backgroundcolor = textfieldbg,%
        width = 3em,%
        height = .8\baselineskip%
    ]{\emph{\textbf{Account:}}}

    \vspace{\sectionSpace}
    \emph{\textbf{Miscellaneous Notes:}}\vspace{3px}\newline
    \TextField[
        name = security.miscnotes,
        multiline = true,
        borderwidth = \bWidth,
        bordercolor = borderorange,
        backgroundcolor = textfieldbg,
        width = \textwidth,
        height = 13\baselineskip
    ]{}
\end{Form}
\pagebreak
\section*{Fire System\blfootnote{\hyperlink{toc}{Back to Top}}}
\rule[10pt]{\textwidth}{2pt}
\addcontentsline{toc}{section}{Fire System}
\label{sec:Fire}

\noindent
\begin{Form}
    \TextField[
        name = fire.paneltype,
        borderwidth = \bWidth,
        bordercolor = borderorange,
        backgroundcolor = textfieldbg,
        width = 7em,
        height = .8\baselineskip
    ]{\textbf{\emph{Panel Type:}}}

    \vspace{\sectionSpace}
    \textbf{\emph{Location:}}\newline
    \vspace{2px}
    \TextField[
        name = fire.panellocation,
        borderwidth = \bWidth,
        bordercolor = borderorange,
        backgroundcolor = textfieldbg,
        width = \textwidth,
        height = .8\baselineskip
    ]{}

    \noindent
    \vspace{\sectionSpace}
    \TextField[
        name = fire.breakernumber,
        borderwidth = \bWidth,
        bordercolor = borderorange,
        backgroundcolor = textfieldbg,
        width = 2.2em,
        height = .8\baselineskip
    ]{\textbf{\emph{Breaker Number:}}}

    \noindent
    \vspace{3px}
    \makeatletter
    \@for%
    \ipbox:={1,2,3,4}%
    \do{%
        \ifnum\ipbox=1%
            \TextField[%
                name = fire.ipbox\ipbox,%
                maxlen = 3,%
                borderwidth = \bWidth,%
                bordercolor = borderorange,%
                backgroundcolor = textfieldbg,%
                width = 2.2em,%
                height = .7\baselineskip%
            ]{\emph{\textbf{IPv4 Address}}}
        \else%
            \TextField[%
                name = fire.ipbox\ipbox,%
                maxlen = 3,%
                borderwidth = \bWidth,%
                bordercolor = borderorange,%
                backgroundcolor = textfieldbg,%
                width = 2.2em,%
                height = .7\baselineskip%
            ]{\emph{\textbf{.}}}
        \fi%
    }%
    \makeatother

    \noindent
    \vspace{\sectionSpace}
    \makeatletter
    \@for%
    \macbox:={1,2,3,4,5,6}%
    \do{%
        \ifnum\macbox=1%
            \TextField[%
                name = fire.macbox\macbox,%
                maxlen = 2,%
                borderwidth = \bWidth,%
                bordercolor = borderorange,%
                backgroundcolor = textfieldbg,%
                width = 2em,%
                height = .8\baselineskip%
            ]{\emph{\textbf{M.A.C:}}}
        \else%
            \TextField[%
                name = fire.macbox\macbox,%
                maxlen = 2,%
                borderwidth = \bWidth,%
                bordercolor = borderorange,%
                backgroundcolor = textfieldbg,%
                width = 2em,%
                height = .8\baselineskip%
            ]{\emph{\textbf{:}}}
        \fi%
    }%
    \makeatother
    \hspace{.1em}
    \TextField[
        name = fire.crc,
        % maxlen = 4,
        borderwidth = \bWidth,
        bordercolor = borderorange,
        backgroundcolor = textfieldbg,
        width = 7.5em,
        height = .8\baselineskip
    ]{\textbf{\emph{C.R.C:}}}

    \noindent
    \makeatletter
    \@for%
    \psulocation:={1,2,3,4}%
    \do{%
        \vspace{3px}%
        \emph{\textbf{Power Supply \#\psulocation\ Location:}}%
        \newline%
        \TextField[%
            name = fire.psulocation\psulocation,%
            borderwidth = \bWidth,%
            bordercolor = borderorange,%
            backgroundcolor = textfieldbg,%
            width = \textwidth,%
            height = .8\baselineskip%
        ]{}%
        \ifnum\psulocation<4{%
            \newline%
        }%
        \fi%
    }%
    \makeatother

    \vspace{\sectionSpace}
    \emph{\textbf{Miscellaneous Notes:}}\vspace{3px}\newline
    \TextField[
        name = fire.miscnotes,
        multiline = true,
        borderwidth = \bWidth,
        bordercolor = borderorange,
        backgroundcolor = textfieldbg,
        width = \textwidth,
        height = 9\baselineskip
    ]{}
\end{Form}
\pagebreak
\section*{Camera System\blfootnote{\hyperlink{toc}{Back to Top}}}
\rule[10pt]{\textwidth}{2pt}
\addcontentsline{toc}{section}{Camera System}
\label{sec:Camera}

\noindent
\begin{Form}
    \TextField[
        name = camera.paneltype,
        borderwidth = \bWidth,
        bordercolor = borderorange,
        backgroundcolor = textfieldbg,
        width = 12em,
        height = .8\baselineskip
    ]{\textbf{\emph{Recorder Type:}}}

    \vspace{\sectionSpace}
    \textbf{\emph{Location:}}\newline
    \vspace{2px}
    \TextField[
        name = camera.panellocation,
        borderwidth = \bWidth,
        bordercolor = borderorange,
        backgroundcolor = textfieldbg,
        width = \textwidth,
        height = .8\baselineskip
    ]{}

    \vspace{3px}
    \footnotesize
    \textcolor{hintgray}{
        (Put in the box below the last 9 characters (could be numbers or letters) that
        come before the ``WCVU" at the end of the recorder's full serial number. An
        example with the correct data highlighted in red will be below.)
        \begin{center}0820160211AAWR\textcolor{hinthighlight}{123456789}WCVU\end{center}
    }

    \normalsize
    \noindent
    \TextField[
        name = camera.serialnumber,
        maxlen = 9,
        borderwidth = \bWidth,
        bordercolor = borderorange,
        backgroundcolor = textfieldbg,
        width = 5em,
        height = .8\baselineskip
    ]{\emph{\textbf{Serial Number:}}}

    \vspace{\sectionSpace}
    \noindent
    \makeatletter
    \@for%
    \ipbox:={1,2,3,4}%
    \do{%
        \ifnum\ipbox=1%
            \TextField[%
                name = camera.wanipbox\ipbox,%
                maxlen = 3,%
                borderwidth = \bWidth,%
                bordercolor = borderorange,%
                backgroundcolor = textfieldbg,%
                width = 2.2em,%
                height = .7\baselineskip%
            ]{\emph{\textbf{Public IPv4 Address}}}
        \else%
            \TextField[%
                name = camera.wanipbox\ipbox,%
                maxlen = 3,%
                borderwidth = \bWidth,%
                bordercolor = borderorange,%
                backgroundcolor = textfieldbg,%
                width = 2.2em,%
                height = .7\baselineskip%
            ]{\emph{\textbf{.}}}
        \fi%
    }%
    \makeatother

    \vspace{3px}
    \noindent
    \makeatletter
    \@for%
    \ipbox:={1,2,3,4}%
    \do{%
        \ifnum\ipbox=1%
            \TextField[%
                name = camera.lanipbox\ipbox,%
                maxlen = 3,%
                borderwidth = \bWidth,%
                bordercolor = borderorange,%
                backgroundcolor = textfieldbg,%
                width = 2.2em,%
                height = .7\baselineskip%
            ]{\emph{\textbf{Pravate IPv4 Address}}}
        \else%
            \TextField[%
                name = camera.lanipbox\ipbox,%
                maxlen = 3,%
                borderwidth = \bWidth,%
                bordercolor = borderorange,%
                backgroundcolor = textfieldbg,%
                width = 2.2em,%
                height = .7\baselineskip%
            ]{\emph{\textbf{.}}}
        \fi%
    }%
    \makeatother

    \vspace{\sectionSpace}
    \noindent
    \makeatletter
    \@for%
    \camsys:={1,2,3}%
    \do{%
    \ifnum\camsys=1{%
        \emph{\textbf{Admin Login:}}%
        }%
        \else{%
            \emph{\textbf{User \#\camsys\ Login:}}%
        }%
    \fi%
    \vspace{1px}%
    \newline%
    \ifnum\camsys<3{%
        \vspace{5px}%
    }%
    \fi%
    \ifnum\camsys=1{%
        \emph{\textbf{User:}}\hspace{4.35em}admin%
        \hspace{4.35em}%
        }%
        \else{%
            \TextField[%
                name = camera.username\camsys,%
                borderwidth = \bWidth,%
                bordercolor = borderorange,%
                backgroundcolor = textfieldbg,%
                width = 10em,%
                height = .8\baselineskip%
            ]%
            {\emph{\textbf{User:}}}%
            \hspace{1em}%
        }%
    \fi%
    \TextField[%
        name = camera.password\camsys,%
        borderwidth = \bWidth,%
        bordercolor = borderorange,%
        backgroundcolor = textfieldbg,%
        width = 10em,%
        height = .8\baselineskip%
    ]%
    {\emph{\textbf{Pass:}}}%
    \newline%
    }%
    \makeatother

    \vspace{\sectionSpace}
    \emph{\textbf{Ports:}}
    \begin{center}
    \makeatletter
    \@for%
    \ports:={1,2,3,4}%
    \do{%
    \TextField[%
    name = camera.port\ports,%
    maxlen = 5,%
    borderwidth = \bWidth,%
    bordercolor = borderorange,%
    backgroundcolor = textfieldbg,%
    width = 3.2em,%
    height = .8\baselineskip%
    ]{\emph{\textbf{%
    \ifnum\ports=1{%
        HTTP:%
        }%
    \else{%
        \ifnum\ports=2{%
            \hspace{.5em}%
            RTSP:%
            }%
        \else{%
            \ifnum\ports=3{%
                \hspace{.5em}%
                Server:%
            }%
            \else{%
                \hspace{.5em}%
                HTTPS:%
            }%
            \fi%
        }%
        \fi%
    }%
    \fi%
    }}}}%
    \end{center}
    \makeatother

    \vspace{\sectionSpace}
    \emph{\textbf{Miscellaneous Notes:}}\vspace{3px}\newline
    \TextField[
        name = camera.miscnotes,
        multiline = true,
        borderwidth = \bWidth,
        bordercolor = borderorange,
        backgroundcolor = textfieldbg,
        width = \textwidth,
        height = 2\baselineskip
    ]{}
\end{Form}
\pagebreak
\section*{Access System\blfootnote{\hyperlink{toc}{Back to Top}}}
\rule[10pt]{\textwidth}{2pt}
\addcontentsline{toc}{section}{Access System}
\label{sec:Access}

\noindent
\begin{Form}
    \TextField[
        name = access.paneltype,
        borderwidth = \bWidth,
        bordercolor = borderorange,
        backgroundcolor = textfieldbg,
        width = 10em,
        height = .8\baselineskip
    ]{\textbf{\emph{Panel Type:}}}
    \TextField[
        name = access.panelid,
        borderwidth = \bWidth,
        bordercolor = borderorange,
        backgroundcolor = textfieldbg,
        width = 10em,
        height = .8\baselineskip
    ]{\textbf{\emph{Panel ID:}}}

    \vspace{\sectionSpace}
    \textbf{\emph{Location:}}\newline
    \vspace{2px}
    \TextField[
        name = access.panellocation,
        borderwidth = \bWidth,
        bordercolor = borderorange,
        backgroundcolor = textfieldbg,
        width = \textwidth,
        height = .8\baselineskip
    ]{}

    \vspace{\sectionSpace}
    \noindent
    \makeatletter
    \@for%
    \macbox:={1,2,3,4,5,6}%
    \do{%
        \ifnum\macbox=1%
            \TextField[%
                name = access.macbox\macbox,%
                maxlen = 2,%
                borderwidth = \bWidth,%
                bordercolor = borderorange,%
                backgroundcolor = textfieldbg,%
                width = 2em,%
                height = .8\baselineskip%
            ]{\emph{\textbf{M.A.C:}}}
        \else%
            \TextField[%
                name = access.macbox\macbox,%
                maxlen = 2,%
                borderwidth = \bWidth,%
                bordercolor = borderorange,%
                backgroundcolor = textfieldbg,%
                width = 2em,%
                height = .8\baselineskip%
            ]{\emph{\textbf{:}}}
        \fi%
    }%
    \makeatother

    \noindent
    \vspace{3px}
    \makeatletter
    \@for%
    \ipbox:={1,2,3,4}%
    \do{%
        \ifnum\ipbox=1%
            \TextField[%
                name = access.ipbox\ipbox,%
                maxlen = 3,%
                borderwidth = \bWidth,%
                bordercolor = borderorange,%
                backgroundcolor = textfieldbg,%
                width = 2.2em,%
                height = .7\baselineskip%
            ]{\emph{\textbf{IPv4 Address}}}
        \else%
            \TextField[%
                name = access.ipbox\ipbox,%
                maxlen = 3,%
                borderwidth = \bWidth,%
                bordercolor = borderorange,%
                backgroundcolor = textfieldbg,%
                width = 2.2em,%
                height = .7\baselineskip%
            ]{\emph{\textbf{.}}}
        \fi%
    }%
    \makeatother

    \vspace{\sectionSpace}
    \noindent
    \TextField[
        name = access.siteid,
        borderwidth = \bWidth,
        bordercolor = borderorange,
        backgroundcolor = textfieldbg,
        width = 6em,
        height = .8\baselineskip
    ]{\textbf{\emph{Site ID:}}}

    \vspace{5px}
    \noindent
    \makeatletter
    \@for%
    \camsys:={1,2,3}%
    \do{%
    \ifnum\camsys=1{%
        \emph{\textbf{Keyscan Login:}}%
        }%
        \else{%
            \emph{\textbf{User \#\camsys\ Login:}}%
        }%
    \fi%
    \vspace{1px}%
    \newline%
    \ifnum\camsys<3{%
        \vspace{5px}%
    }%
    \fi%
    \ifnum\camsys=1{%
        \emph{\textbf{User:}}\hspace{3.25em}KEYSCAN%
        \hspace{3.25em}%
        }%
        \else{%
            \TextField[%
                name = access.username\camsys,%
                borderwidth = \bWidth,%
                bordercolor = borderorange,%
                backgroundcolor = textfieldbg,%
                width = 10em,%
                height = .8\baselineskip%
            ]%
            {\emph{\textbf{User:}}}%
            \hspace{1em}%
        }%
    \fi%
    \TextField[%
        name = access.password\camsys,%
        borderwidth = \bWidth,%
        bordercolor = borderorange,%
        backgroundcolor = textfieldbg,%
        width = 10em,%
        height = .8\baselineskip%
    ]%
    {\emph{\textbf{Pass:}}}%
    \newline%
    }%
    \makeatother

    \vspace{\sectionSpace}
    \emph{\textbf{Miscellaneous Notes:}}\vspace{3px}\newline
    \TextField[
        name = access.miscnotes,
        multiline = true,
        borderwidth = \bWidth,
        bordercolor = borderorange,
        backgroundcolor = textfieldbg,
        width = \textwidth,
        height = 10\baselineskip
    ]{}
\end{Form}
\end{document}